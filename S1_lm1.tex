
\documentclass[12pt]{article}              
\usepackage[margin=2.0cm]{geometry}
\usepackage{setspace,relsize}               
\usepackage{moreverb}                        
\usepackage{url}
\usepackage{hyperref}
\hypersetup{colorlinks=true,citecolor=blue}
\usepackage{amsmath}
\usepackage{mathtools} 
\usepackage{amssymb}
\usepackage{indentfirst}
\usepackage[authoryear,round]{natbib}
\bibliographystyle{apalike}
\usepackage[pdftex]{lscape}
\usepackage[toc,page]{appendix}
% \usepackage{float}
% \usepackage{longtable}
% \usetikzlibrary{arrows}
%%%%
\title{
Supporting Information to Estimating the Initial Susceptible Fraction in Dengue Dynamics"
}           
\author{
Flavio Code\c{c}o Coelho \\
\\
\& \\
Luiz Max de Carvalho \\
}

\date{}
% Notation defs
\def \rr {$R_{t}\ $}

%\usepackage{Sweave}
\begin{document}                                  
% \input{draft_lm1-concordance}

%
\maketitle

\subsection*{A remark on prior distributions and tail behavior of the 
distribution of $R_t$}
\label{sec:tails}

In order to decide which approach to take, it may be of use analysing the variance and tail behaviour of the derived distributions for \rr. 
Consider the case of using a flat $Uniform(0, 1)$ prior for $\theta_t$.
With $a_0 = b_0 = 1$, $a_1 = a_2$ and $b_1 = b_2 + 1$.
The beta prime (inverse beta distribution) will have heavier tails compared to the conditional distribution in (\ref{eq:densityNishiura}), thus providing more conservative confidence/credibility intervals.  
As a side note, the Bayesian approach presented in Section~\ref{sec:mantel} will give similar results to those of \citet{wilson} and \citet{wilson} for $Y_{t+1}$ and $Y_t >> 1$.
Under the flat  uniform prior for $\theta_t$, the Bayesian posterior credibility interval is nearly indistinguishable from the confidence interval proposed by \citet{clopper} for $Y_{t+1}, Y_t > 20$.
Note that the $Beta(1, 1)$ uniform prior for $\theta_t$ constitutes a poor prior choice mainly because the induced distribution for \rr is only well-defined for $b_0 > 2$.

The main advange of the Bayesian approach is that one can devise prior distributions for $\theta_t$ taking advantage of the intuitive parametrisation and flexibility of the beta family of distributions.
Prior elicitation can also be done for \rr and the hyperparameters directly plugged into the prior for $\theta_t$. 
One can, for example, choose a priori mean and variance for \rr and find $a_0$ and $b_0$ that satisfy those conditions.
Let $m_0$ and $v_0$ be the prior expectation and variance for $R_t$. 
After some tedious algebra one finds
\begin{align}
\label{eq:elicitation}
a_0 &= \frac{m_0v_0 + m_0^3 + m_0^2}{v_0} \\
b_0 &= \frac{2v_0 + m_0^2 + m_0}{v_0}
\end{align}
If one wants only to specify $m_0$ and a coefficient of variation $c$ \footnote{$c = \sqrt{v_0}/ m_0$.} for $R_t$ \textit{a priori}, some less boring algebra gives:
\begin{align}
\label{eq:elicitationcv}
a_0 &= \frac{m_0^3c^2 + m_0^3 + m_0^2}{m_0^2c^2} \\
b_0 &= \frac{2m_0^2c^2 + m^2 + m}{m_0^2c^2}
\end{align}
\newpage
\bibliography{lm1}
\end{document}
