\documentclass[a4paper,10pt]{article}
\usepackage[utf8]{inputenc}
\usepackage[numbers]{natbib}
\bibliographystyle{naturemag}
% amsmath package, useful for mathematical formulas
\usepackage{amsmath}
% amssymb package, useful for mathematical symbols
\usepackage{amssymb}
\usepackage{url}
\usepackage{graphicx}
% Our defs 
\def \rr {$R_{t}\:$}
%opening
\title{Supplementary material to ``Estimating the Attack Ratio of Dengue 
Epidemics under Time-varying 
Force of Infection using Aggregated Notification Data''}
\author{Flavio Code\c{c}o Coelho and Luiz Max de Carvalho}

\begin{document}

\maketitle

\section*{A remark on prior distributions and tail behaviour of the 
distribution of $R_t$}
\label{sec:tails}
There are a number of approaches to deriving the distribution of~\rr.
Alternatively to the approach described in the main text~\citep{mantel}, one 
could use the conditional distribution of \rr on 
$Y_{t+1}$ and $Y_t$ as defined in equation A7 of Nishiura et 
al.~\citep{nishiura}:
\begin{equation}
\label{seq:unorm}
f_{R}(R_{t}) = (Y_tR_{t})^{Y_{t+1}} e^{-Y_tR_{t}}
\end{equation}
Noticing the kernel of (\ref{seq:unorm}) is that of a gamma distribution with 
$a_2 = Y_{t+1}+1$ and $b_2 = Y_t$, we obtain a proper density from which to 
construct $c_{\alpha}(R_t)$, simply by computing the appropriate quantiles of 
said distribution.
 This density is
\begin{equation}
\label{seq:densityNishiura}
f_N(R_t| a_2, b_2) =  \frac{b_2^{a_2}}{\Gamma(a_2)} R_t^{a_2-1} e^{-b_2 R_t}
\end{equation}

In order to decide which approach to take, it may be of use analysing the 
tail behaviour of the derived distributions for \rr. 
Consider the case of using a flat $\text{Uniform}(0, 1)$ prior for $\theta_t$.
With $a_0 = b_0 = 1$, $a_1 = a_2$ and $b_1 = b_2 + 1$.
The beta prime (inverse beta distribution) will have heavier tails compared to 
the conditional distribution proposed by~\citep{nishiura}, thus providing more 
conservative confidence/credibility intervals.
To see that one needs simply take the ratio of the Beta prime and Gamma 
(unnormalized) densities and evaluate the limit as $R_t$ goes to infinity:
\begin{equation}
 \label{seq:densityratio}
 \lim_{R_t\to\infty}\frac{f_P(R_t| a_1, b_1)}{f_N(R_t| a_2, b_2)} =  
\lim_{R_t\to\infty}\frac{e^{Y_{t}R_t}}{(1 +R_t)^{Y_{t} + Y_{t +1}+2}} = \infty
\end{equation}
Note also that we deliberately construct $c_{\alpha}(R_{t})$ as a 
equal-tailed $100\alpha\%$ credible set, rather than a less conservative 
highest posterior density (HPD) interval.
Finally, we performed a simple simulation study to assess the result in equation
~\ref{seq:densityratio}.
The simulations were carried out as follows:
\begin{enumerate}
 \item Create a two-dimensional grid of values for $\lambda_1$ and $\lambda_2$, 
with rate values from $2$ to $1000$;
\item For each point $(\lambda_1^{(j)}, \lambda_2^{(j)})$ in the grid,
\begin{itemize}
 \item Generate $1000$ realisations of the random variables $Y_{1}^{(ij)} \sim 
\text{Poisson}(\lambda_1^{(j)})$ and $Y_{2}^{(ij)} \sim 
\text{Poisson}(\lambda_2^{(j)})$, with $i= 1, 2, \ldots, 1000$  and compute 
$R_{t}^{(ij)} = Y_{2}^{(ij)}/
Y_{1}^{(ij)}$;
\item Compute the $\alpha\%$ credibility/confidence intervals using the method 
proposed here $(\beta)$ and using the distribution proposed by Nishiura 
(2010)~\cite{nishiura} ($\Gamma$), denoted by $c_{\beta}(R_{t}^{(ij)}; 
\alpha)$ and 
$c_{\Gamma}(R_{t}^{(ij)}; \alpha)$ respectively;
\item Determine whether
\begin{itemize}
 \item [(i)] $ c_{\Gamma}(R_{t}^{(ij)}; \alpha) \subseteq 
c_{\beta}(R_{t}^{(ij)}; \alpha)$;
\item [(ii)] $c_{\beta}(R_{t}^{(ij)}; \alpha) \subseteq 
c_{\Gamma}(R_{t}^{(ij)}; \alpha)$ or;
\item [(iii)] there is overlap
\end{itemize}
\item Calculate $Pr(R_{t}^{(ij)} > 1)$ using both distributions
\end{itemize}
\end{enumerate}
The results show that the credibility intervals obtained using the distribution 
proposed here were larger than (i.e. contained) those computed using the 
distribution in~\ref{seq:densityNishiura} in $90.69\%$ of the simulations 
(integrating over the grid), while the converse was observed in $1.14\%$ of the 
cases.
Moreover, our method yielded lower $Pr(R_{t} > 1)$  in $79.48\%$ of the 
simulations, indicating it is indeed more robust to false alarm.
An R script to perform the simulation study described above can be found 
at~\url{
https://github.com/fccoelho/paperLM1/blob/master/R/credibility_comparison.R}.

As a side note, the Bayesian approach presented in this 
paper will give similar results to orthodox confidence intervals~\citep{wilson} 
and~\citep{clopper} for $Y_{t+1}$ and $Y_t >> 1$.
Under the flat  uniform prior for $\theta_t$, the Bayesian posterior 
credibility interval is nearly indistinguishable from the confidence interval 
proposed by Clopper \& Pearson (1931)~\citep{clopper} for $Y_{t+1}, Y_t > 20$.
Note also that the uniform prior ($Beta(1, 1)$) for $\theta_t$ constitutes a 
poor prior choice mainly because the induced distribution for \rr is only 
well-defined for $b_0 > 2$.

An advantage of the Bayesian approach is that one can devise prior 
distributions for $\theta_t$ taking advantage of the intuitive parametrization 
and flexibility of the beta family of distributions.
Prior elicitation can also be done for~\rr and the hyper-parameters directly 
plugged into the prior for $\theta_t$. 
One can, for example, choose prior mean and variance for~\rr and find $a_0$ 
and $b_0$ that satisfy those conditions.
Let $m_0$ and $v_0$ be the prior expectation and variance for $R_t$. 
After some tedious algebra one finds
\begin{align}
\label{seq:elicitation}
a_0 &= \frac{m_0v_0 + m_0^3 + m_0^2}{v_0} \\
b_0 &= \frac{2v_0 + m_0^2 + m_0}{v_0}
\end{align}
If one wants only to specify $m_0$ and the coefficient of variation $c = 
\sqrt{v_0}/ m_0$ for $R_t$ \textit{a priori}, some less boring 
algebra gives:
\begin{align}
\label{seq:elicitationcv}
a_0 &= \frac{m_0^3c^2 + m_0^3 + m_0^2}{m_0^2c^2} \\
b_0 &= \frac{2m_0^2c^2 + m^2 + m}{m_0^2c^2}
\end{align}

This approach thus makes it possible to incorporate epidemiological knowledge 
about disease Biology (e.g. the magnitude of $R_0$) into the computation of \rr.
This may prove particularly important when disease counts are low and/or close 
to the detection threshold.
We provide an R script to perform the elicitation described above at 
\url{https://github.com/fccoelho/paperLM1/blob/master/R/elicit_Rt_prior.R}.

\section*{Estimating $S_0$ from simulated data}

In order to determine whether the inference methodology proposed can 
recover the true parameter values of the underlying simulation model, we have 
devised a simple simulation experiment.
Using the SIR model presented in the main paper we have simulated the incidence 
curve of the  2012-13 epidemic, using 
$S_0 = 0.0621$ and \rr, as estimated from actual incidence data, and $\tau=1$. 
Figure~\ref{fig:sim_data} shows the posterior $S$ and $I$ alongside with 
simulated incidence data.
It is clear that the method can recover the correct value for $S_0$ 
used to   simulate incidence.
The script to generate the simulated data and run the inference is available at 
the paper's Github repository 
(\url{
https://github.com/fccoelho/paperLM1/blob/master/python/fit_simulated_data.py}).

\begin{figure}[!ht]
 \centering
 \includegraphics[width=14cm]{./plots/Sim_DengueS2012_11_series.png}
 % Sim_DengueS2012_11_series.png: 0x0 pixel, 300dpi, 0.00x0.00 cm, bb=
 \caption{$S$ and $I$ series fitted to simulated incidence data (blue dots) 
with $S_0=0.0621$, and \rr estimated from real incidence data.}
 \label{fig:sim_data}
\end{figure}

\subsection*{DiffeRential Evolution Adaptive Metropolis (DREAM)}

Since the posterior distributions desired in this paper are not available in 
closed-form, we need to resort to numerical methods to obtain approximations.
We employ an adaptive Markov chain Monte Carlo (MCMC) algorithm, proposed
by~\cite{vrugt2008}, called  DiffeRential Evolution Adaptive Metropolis (DREAM).

DREAM draws on the basic structure of Differential Evolution Markov Chain 
(DE-MC) which runs $N$ independent chains in parallel and accepts moves 
proportional to the difference of two randomly sampled members (chains), thus 
differentially evolving conditional on the proposal variances.
An additional aspect of DREAM is the so-called delayed rejection.
This procedure adapts the original Metropolis-Hastings algorithm by attempting 
new moves whenever a proposal is rejected (instead of setting the current state 
to the previously accepted one), thus delaying the rejection of proposals. 
This potentially decreases autocorrelation and improves mixing 
(see~\cite{mira2001} for details). 

A desirable property of DREAM is that one can scale the algorithm with model 
complexity, meaning one can initialise as many parallel chains as there are 
parameters in the model, thus exploiting the multi-core architecture of modern 
computers to speed up convergence of the MCMC.

A Python implementation of DREAM for dynamic models is available in the package 
Bayesian inference with Python (BIP)~\cite{pone2011} available at 
\url{http://code.google.com/p/bayesian-inference/}.
The code to produce the results presented in this paper is available 
at~\url{https://github.com/fccoelho/paperLM1/tree/master/python}.

\begin{thebibliography}{1}
\expandafter\ifx\csname url\endcsname\relax
  \def\url#1{\texttt{#1}}\fi
\expandafter\ifx\csname urlprefix\endcsname\relax\def\urlprefix{URL }\fi
\providecommand{\bibinfo}[2]{#2}
\providecommand{\eprint}[2][]{\url{#2}}

\bibitem{mantel}
\bibinfo{author}{Ederer, F.} \& \bibinfo{author}{Mantel, N.}
\newblock \bibinfo{title}{Confidence limits on the ratio of two poisson
  variables}.
\newblock \emph{\bibinfo{journal}{American Journal of Epidemiology}}
  \textbf{\bibinfo{volume}{100}}, \bibinfo{pages}{165--167}
  (\bibinfo{year}{1974}).

\bibitem{nishiura}
\bibinfo{author}{Nishiura, H.}, \bibinfo{author}{Chowell, G.},
  \bibinfo{author}{Heesterbeek, H.} \& \bibinfo{author}{Wallinga, J.}
\newblock \bibinfo{title}{{{T}he ideal reporting interval for an epidemic to
  objectively interpret the epidemiological time course}}.
\newblock \emph{\bibinfo{journal}{J R Soc Interface}}
  \textbf{\bibinfo{volume}{7}}, \bibinfo{pages}{297--307}
  (\bibinfo{year}{2010}).

\bibitem{wilson}
\bibinfo{author}{Wilson, E.~B.}
\newblock \bibinfo{title}{Probable inference, the law of succession, and
  statistical inference}.
\newblock \emph{\bibinfo{journal}{Journal of the American Statistical
  Association}} \textbf{\bibinfo{volume}{22}}, \bibinfo{pages}{209--212}
  (\bibinfo{year}{1927}).

\bibitem{clopper}
\bibinfo{author}{Clopper, C.} \& \bibinfo{author}{Pearson, E.~S.}
\newblock \bibinfo{title}{The use of confidence or fiducial limits illustrated
  in the case of the binomial}.
\newblock \emph{\bibinfo{journal}{Biometrika}} \bibinfo{pages}{404--413}
  (\bibinfo{year}{1934}).
  
\bibitem{vrugt2008}
\bibinfo{author}{Vrugt, J.~A.} \emph{et~al.}
\newblock \bibinfo{title}{Accelerating markov chain monte carlo simulation by
  differential evolution with self-adaptive randomized subspace sampling}.
\newblock \emph{\bibinfo{journal}{International Journal of Nonlinear Sciences
  and Numerical Simulation}} \textbf{\bibinfo{volume}{10}},
  \bibinfo{pages}{271--288} (\bibinfo{year}{2008}).

  
\bibitem{mira2001}
\bibinfo{author}{Mira, A.}
\newblock \bibinfo{title}{On metropolis-hastings algorithms with delayed
  rejection}.
\newblock \emph{\bibinfo{journal}{Metron}} \textbf{\bibinfo{volume}{59}},
  \bibinfo{pages}{231--241} (\bibinfo{year}{2001}).
  
\bibitem{pone2011}
\bibinfo{author}{Coelho, F.~C.}, \bibinfo{author}{Code\c{c}o, C.~T.} \&
  \bibinfo{author}{Gomes, M.~G.}
\newblock \bibinfo{title}{{A} {B}ayesian framework for parameter estimation in
  dynamical models}.
\newblock \emph{\bibinfo{journal}{PLoS ONE}} \textbf{\bibinfo{volume}{6}},
  \bibinfo{pages}{e19616} (\bibinfo{year}{2011}).

\end{thebibliography}


\end{document}
