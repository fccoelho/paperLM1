\documentclass[a4paper,10pt]{letter}
\usepackage[utf8]{inputenc}

\begin{document}
% If you want headings on subsequent pages,
% remove the ``%'' on the next line:
% \pagestyle{headings}

\begin{letter}{Laura White\\Editorial Board Member\\Scientific Reports }
\address{Escola de Matem\'atica Aplicada\\Funda\c{c}\~ao Getulio Vargas 
(FGV)\\Rio de Janeiro -- RJ\\Brazil.}

\opening{Dear Editor,}

In response to the comments of the reviewers on our manuscript, we have made 
the following modifications to the manuscript:

\begin{description}
 \item[Reviewer 1:] Responses to the main comments are listed below. The minor 
comments where also integrally accepted, namely the  improvement of the README 
in the code repository as well as the citation of two  additional papers. 
 
 \begin{enumerate}
  \item The response to this comment is two part: first, we performed 
a simple study to assess whether our method of estimation of $R_{t}$ provides 
more conservative credibility intervals and thus offers protection against 
false alarm.
In agreement with the analytical result shown in equation (16), 
our simulation study -- described at the end of the section on 
the  tail behaviour of the distribution of $R_t$ section -- shows that our 
method provided wider CIs in $\approx 90\%$ of the time.
Then, in order to assess the ability to recover the parameters, simulation 
experiment was done, as requested and included as a new section in the  
supplementary material. The simulations shows that the inference 
methodology can recover the exact parameters used to simulate  the data.
\item The text of the methods section was improved to answer the four issues 
raised in this comment.
\item The Introduction and discussion sections were expanded to included the 
additional information required by this comment.
 \end{enumerate}

 \item [Reviewer 2:] \textbf{Minor comments:} Fixed typo in the abstract; Added 
text to the discussion stating that we expect the effects of underreporting to 
cancel out in the calculation of the attack ratio.
Moreover, a section explaining the details of DREAM (delayed rejection, 
adaptative MCMC, etc.) was added to the Appendix.
 \begin{enumerate}
  \item Reference 5, despite having similar concepts as those discussed in our 
paper, has quite different goals. we have however, cited another publication 
(ref 26), which is from the same group and based on the same field study, and 
use their findings about seroprevalence in the 'results and discussion' section.
\item We have not found articles combining the use of aggregated data to study 
variable force of infection, however we have reviewed other works modeling 
variable FoI (refs 7, 8, 9, 10, 12).
 \end{enumerate}

\end{description}

Having fulfilled all the demands of the reviewers we re-submit the modified 
manuscript and await the final decision. As always, please do not hesitate 
to contact us if further questions arise.


\signature{Flávio Codeço Coelho\\Professor}

\closing{Sincerely}

%enclosure listing
%\encl{}

\end{letter}
\end{document}
