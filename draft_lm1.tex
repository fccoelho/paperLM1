% Template for PLoS
% Version 2.0 July 2014
%
% To compile to pdf, run:
% latex plos.template
% bibtex plos.template
% latex plos.template
% latex plos.template
% dvipdf plos.template
%
% % % % % % % % % % % % % % % % % % % % % %
%
% -- IMPORTANT NOTE
%
% Be advised that this is merely a template 
% designed to facilitate accurate translation of manuscript content 
% into our production files. 
%
% This template contains extensive comments intended 
% to minimize problems and delays during our production 
% process. Please follow the template 
% whenever possible.
%
% % % % % % % % % % % % % % % % % % % % % % % 
%
% Once your paper is accepted for publication and enters production, 
% PLEASE REMOVE ALL TRACKED CHANGES in this file and leave only
% the final text of your manuscript.
%
% DO NOT ADD EXTRA PACKAGES TO THIS TEMPLATE unless absolutely necessary.
% Packages included in this template are intentionally
% limited and basic in order to reduce the possibility
% of issues during our production process.
%
% % % % % % % % % % % % % % % % % % % % % % %
%
% -- FIGURES AND TABLES
%
% DO NOT INCLUDE GRAPHICS IN YOUR MANUSCRIPT
% - Figures should be uploaded separately from your manuscript file. 
% - Figures generated using LaTeX should be extracted and removed from the PDF before submission. 
% - Figures containing multiple panels/subfigures must be combined into one image file before submission.
% See http://www.plosone.org/static/figureGuidelines for PLOS figure guidelines.
%
% Tables should be cell-based and may not contain:
% - tabs/spacing/line breaks within cells to alter layout
% - vertically-merged cells (no tabular environments within tabular environments, do not use \multirow)
% - colors, shading, or graphic objects
% See http://www.plosone.org/static/figureGuidelines#tables for table guidelines.
%
% For sideways tables, use the {rotating} package and use \begin{sidewaystable} instead of \begin{table} in the appropriate section. PLOS guidelines do not accomodate sideways figures.
%
% % % % % % % % % % % % % % % % % % % % % % % %
%
% -- EQUATIONS, MATH SYMBOLS, SUBSCRIPTS, AND SUPERSCRIPTS
%
% IMPORTANT
% Below are a few tips to help format your equations and other special characters according to our specifications. For more tips to help reduce the possibility of formatting errors during conversion, please see our LaTeX guidelines at http://www.plosone.org/static/latexGuidelines
%
% Please be sure to include all portions of an equation in the math environment, and for any superscripts or subscripts also include the base number/text. For example, use $mathrm{mm}^2$ instead of mm$^2$ (do not use \textsuperscript command).
%
% DO NOT USE the \rm command to render mathmode characters in roman font, instead use $\mathrm{}$
% For bolding characters in mathmode, please use $\mathbf{}$ 
%
% Please add line breaks to long equations when possible in order to fit our 2-column layout. 
%
% For inline equations, please do not include punctuation within the math environment unless this is part of the equation.
%
% For spaces within the math environment please use the \; or \: commands, even within \text{} (do not use smaller spacing as this does not convert well).
%
%
% % % % % % % % % % % % % % % % % % % % % % % %

\documentclass[10pt]{article}
% amsmath package, useful for mathematical formulas
\usepackage{amsmath}
% amssymb package, useful for mathematical symbols
\usepackage{amssymb}
% cite package, to clean up citations in the main text. Do not remove.
\usepackage{cite}
\usepackage{hyperref}
% line numbers
\usepackage{lineno}
% ligatures disabled
\usepackage{microtype}
\usepackage{todonotes}
\DisableLigatures[f]{encoding = *, family = * }
% rotating package for sideways tables
%\usepackage{rotating}
% If you wish to include algorithms, please use one of the packages below. Also, please see the algorithm section of our LaTeX guidelines (http://www.plosone.org/static/latexGuidelines) for important information about required formatting.
%\usepackage{algorithmic}
%\usepackage{algorithmicx}
% Use doublespacing - comment out for single spacing
%\usepackage{setspace} 
%\doublespacing
% Text layout
\topmargin 0.0cm
\oddsidemargin 0.5cm
\evensidemargin 0.5cm
\textwidth 16cm 
\textheight 21cm
% Bold the 'Figure #' in the caption and separate it with a period
% Captions will be left justified
\usepackage[labelfont=bf,labelsep=period,justification=raggedright]{caption}
% Use the PLoS provided BiBTeX style
\bibliographystyle{plos2009}
% Remove brackets from numbering in List of References
\makeatletter
\renewcommand{\@biblabel}[1]{\quad#1.}
\makeatother


% Leave date blank
\date{}

\pagestyle{myheadings}

%% Include all macros below. Please limit the use of macros.

%% END MACROS SECTION


\begin{document}


% Title must be 150 characters or less
\begin{flushleft}
{\Large
\textbf{Estimating the Initial Susceptible Fraction in Dengue Dynamics}
}
% Insert Author names, affiliations and corresponding author email.
\\
Flavio Code\c{c}o Coelho$^{1\ast}$, 
Luiz Max de Carvalho$^{2}$, 
\\
\bf{1} Escola de Matem\'atica Aplicada , Funda\c{c}\~ao Getulio Vargas (FGV), 
Rio de Janeiro -- RJ, Brazil.
\\
\bf{2} Programa de Computa\c{c}\~ao Cient\'ifica (PROCC), Funda\c{c}\~ao Oswaldo Cruz, Rio de Janeiro -- RJ, Brazil.
\\
$\ast$ E-mail: fccoelho@fgv.br
\end{flushleft}

% Please keep the abstract between 250 and 300 words
\section*{Abstract}
Our paper is just awesome.
% Please keep the Author Summary between 150 and 200 words
% Use first person. PLOS ONE authors please skip this step. 
% Author Summary not valid for PLOS ONE submissions.   
\section*{Author Summary}

\section*{Introduction}
When modeling infecctious disease spread using dynamic models, the initial 
conditions of the system, although key to match the simulated dynamics to 
observed data, are frequently overlooked.
If the model is about a disease 
invading a new population one can simply assume the vast majority of the 
population to be susceptible.
However for a disease which is well in its way to 
endemicity but not yet stable, it can be very hard to determine the 
immunological structure of the population without resorting to expensive large 
scale prevalence tests.

In the particular case of Dengue fever in Brazil, with four circulating 
serotypes, it can be very challenging to parameterize a simulation starting at 
any time other than when the date the disease was reintroduced in the country 
in the late 1980s.
For SIR-like models, the relevant initial values are the 
number of infectious and susceptible  individuals. 
The number of infectious 
individuals are less challenging to determine, because the get very small 
during the winters and can be derived from incidence data  with is monitored 
continuously.
The real challenge is to determine the number of susceptibles to 
a particular strain right at the beginning of the dengue season.

Multiple factors contribute to the dificulty in estimating the number of 
susceptibles in the population, first and foremost is the fact that dengue 

confers permanente immunity within serotypes but with only short-lived 
cross-immunity. Secondly, there are a larger number of asymptomatic cases in 
any epidemic, which nevertheless acquire immunity, Thirdly, under-reporting is 
a serious issue in Brazil. Duarte and Franca\cite{duarte_data_2006}, estimated 
the sensitivity of dengue reporting in Belo-Horizonte, Brazil to be of 63\%, 
meaning that aproximately 37\% of the cases go unreported. Lastly, demography 
and migrations affect the number of susceptible in ways which are not easy to
determine.

The initial number of susceptibles -- $S_0$ -- is a a very powerful parameter 
with regard to epidemic risk.
Its value affects directly the local $R_0$, 
thus being intrumental in predicting epidemics, but it also affects the speed 
and size of the epidemic.

In this paper, we estimate the number of susceptibles, allowing for 
discontinuities right before every epidemic in the last 18 years. THis way we 
can account for replacement of the predominant serotype, which can redefine 
susceptibility for the following season.

To accomplish this we use a simplified model of 
dengue transmission based on a Susceptible-infecctious-Removed (SIR) model. 
Since Dengue is not caused by a single pathogen, but by four separate strains 
with limited cross-immunity, when we talk about number of susceptibles, in the 
context of a single serotype SIR model, we are actually referring to the sum of 
the susceptibles to the currently circulating serotypes at any given time. 

Some information is available about the predominant serotypes during the 
period of study\cite{macedo_virological_2013}. 

The estimation of the  number of susceptibles have been attempted 
before, for other diseases\cite{bjornstad_dynamics_2002, 
wallinga_reconstruction_2003}. These methods try to reconstruct the entire 
series of infectious and susceptibles for measles 
outbreaks from case data.\todo[inline]{anyone else?} In the case of Dengue the 
series of susceptibles to all possible serotypes, cannot be reconstructed based 
solely on a deterministic transmission model, since the arrival/re-emergenge of 
new serotypes (which are a stochastic events) can change drastically the pool 
of susceptibles throwing off any sequential estimation of the dynamics. 
Adopting a multi-strain transmission model, does not help because of 
insufficient information about the proportions of each serotype in case data.

The authors have applied similar methods as used here, to estimate the number 
of susceptibles for the predominant strain of influenza at the beginning of  
influenza seasons in europe\cite{pone2011}.


\section*{Methods}
\subsection*{The data}

The data used to fit the model consists of weekly incidence from 1996 to 2014 
in the city of Rio de Janeiro, Brazil.
From this series, we calculated the effective reproduction number, $R_t$, according to the expression given by Nishiura et al.~\cite{nishiura}.
The method assumes the disease counts $Y(t)$ are Poisson random variables such that $R_t$ is 
\[ R_t = \left(\frac{Y(t + 1)}{Y(t)}\right)^{1/n}\]
where $n$ is the ratio between the length of reporting interval and the mean generation time of the disease.
In this paper we obtain the estimates of $R_t$ using a moving window of three weeks, which roughly corresponds to the generation time of dengue, and thus $n = 1$.
We provide more detail on the estimation and confidence/credibility intervals in Text S1.

\subsection*{The model}

A Susceptible-Infectious-Removed (SIR) model is proposed to model dengue dynamics.
In the traditional formulation of the model, transmission is governed by a transmission rate $\beta$ and recovery happens at a rate $\tau$.
As the epidemic progresses, the effective transmission  rate changes and is
\begin{equation}
 \label{eq:effbeta}
 \beta(t) = R_t	\cdot\tau
\end{equation}

We parameterize our model with the varying transmission using the system of ordinary differential equations
\begin{align}
 \frac{dS}{dt} &= -\beta(t)SI \\
 \frac{dI}{dt} &= \beta(t)SI - \tau I&\\
 \frac{dR}{dt} &= \tau I&
\end{align}
where $S + I + R = 1 \: \forall\: t$. % \in [T_0, T_1]$. 
This is a rather simplified model, in which, for instance, the vector is left out.
Also, although there probably are multiple circulating serotypes, our approach does not discriminate between them.

\subsection*{Bayesian parameter estimation}
For the fitting procedure the time series were normalized to lie on the $[0,1]$ interval, therefore avoiding scaling issues.
% Results and Discussion can be combined.xx
\section*{Results}
% We only support three levels of headings, please do not create a heading level below \subsubsection.
%\subsection*{Subsection 1}
\section*{Discussion}
% Do NOT remove this, even if you are not including acknowledgments.
\section*{Acknowledgments}
LMC is grateful to Dr. Leonardo Bastos for useful discussions on the estimation of $R(t)$.

\bibliography{lm1}
\section*{Figure Legends}
% This section is for figure legends only, do not include
% graphics in your manuscript file.
%
%\begin{figure}
%\caption{
%{\bf Bold the first sentence.}  Rest of figure caption.  
%}
%\label{Figure_label}
%\end{figure}


\section*{Tables}
% 
% See introductory notes if you wish to include sideways tables.
%
% NOTE: Please look over our table guidelines at http://www.plosone.org/static/figureGuidelines#tables to make sure that your tables meet our requirements. Certain types of spacing, cell merging, and other formatting tricks may have unintended results and will be returned for revision.
%
%\begin{table}[!ht]
%\caption{
%\bf{Table title}}
%\begin{tabular}{|c|c|c|}
%table information
%\end{tabular}
%\begin{flushleft}Table caption
%\end{flushleft}
%\label{tab:label}
% \end{table}

\section*{Supporting Information Legends}
%
% Please enter your Supporting Information captions below in the following format:
%\item{\bf Figure SX. Enter mandatory title here.} Enter optional descriptive information here.
% 
%\begin{description}
%\item {\bf}
%\item {\bf}
%\end{description}

\end{document}

